\begin{enumerate}[label=\arabic*.,ref=\theenumi]
	\item The following instructions are for the picow toolchain
		in termux-debian
		\label{it:picow/installation/toolchain}
\begin{lstlisting}
cd .platformio/packages
git clone https://github.com/earlephilhower/arduino-pico
cd arduino-pico
git submodule update --init
cd pico-sdk
git submodule update --init
cd ../tools
python3 ./get.py
\end{lstlisting}
\item Downlaod EtchDroid from F-DROID. 
\item Now execute the following 
\label{it:picow/installation/toolchain/rpi}
\begin{lstlisting}
cd picow/installation/codes/
pio run
cp .pio/build/rpipicow/firmware.uf2 /sdcard/Download/
\end{lstlisting}
\item  Keep BOOTSEL pressed on picow  and connect it to your phone.
\item Now flash picow using EtchDroid.
\item The Onboard LED will start blinking.
\item Change the blink frequency.
\item If Item \ref{it:picow/installation/toolchain} is unsuccessful, 
\begin{enumerate}
	\item
		Execute Item
\ref{it:picow/installation/toolchain/rpi}
on your rpi
	\item Copy relevant platformio packages from your raspberry pi to your phone .platformio/packages directory using scp.
\end{enumerate}
\item 
	See
			\figref{fig:picowpinout} for the picow pin diagram.
			You may reuse all the Arduino codes for pico by changing only the platformio.ini file.
% Insert your image here
		\begin{figure}
			\centering
    \includegraphics[width=\columnwidth]{picow/installation/figs/picowpinout.jpg}
    \caption{picow Pin Diagram} % Adds a centered caption under the image
			\label{fig:picowpinout}
		\end{figure}
%\item Connect RUN on pico W to GND. Keep pressing BOOTSEL while removing the RUN-GND wire from GND.Pico W is now ready to be flashed.
\end{enumerate}





